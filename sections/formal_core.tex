\section{Formal Core}
\label{sec:formal-core}

\subsection{Definitions}

\begin{definition}[Primary Admissible Dynamics]
Let $\mathcal{H}$ denote the set of admissible histories (trajectories) of a system,
determined by fixed constraints or laws. Elements $h \in \mathcal{H}$ are taken as
ontologically primary. No preferred parametrization, ordering, or temporal notion is
assumed at this level.
\end{definition}

\begin{definition}[Parametrization]
A parametrization is a mapping $\lambda: h \mapsto \lambda(h)$ that assigns an ordering
coordinate to points along a history $h$. Different parametrizations (temporal, spatial,
affine, relational, etc.) are coordinate choices on the same admissible structure. No
parametrization is privileged \emph{a priori}.
\end{definition}

\begin{definition}[Parameter Structure]
A parametrization $\lambda$ induces a parameter space $\mathcal{S}$ equipped with
sufficient structure to define separation from a reference value $\lambda_B$ (e.g.\ a
metric, or a topology plus ordering relation sufficient to define neighborhoods and
separation). No physical interpretation of $\lambda$ or $\mathcal{S}$ is assumed.
\end{definition}

\noindent
In what follows, we use a separation function $\operatorname{sep}:\mathcal{S}\times\mathcal{S}\to \mathbb{R}_{\ge 0}$ with minimal properties:
$\operatorname{sep}(x,y)=\operatorname{sep}(y,x)$ and $\operatorname{sep}(x,x)=0$.
No metric structure is assumed unless specified by the application.
The separation function is used only to represent unsigned separation from the conditioning boundary; it is not assumed to encode causal order, temporal flow, or a privileged direction.

\begin{definition}[Time-Reversal Symmetry of the Admissible Structure]
The admissible structure $\mathcal{H}$ is time-reversal symmetric if there exists an
involutive map $\mathcal{R}: \mathcal{H}\rightarrow\mathcal{H}$ such that for every
admissible history $h$, the reversed history $\mathcal{R}(h)$ is also admissible.
This symmetry is local and does not presuppose global cosmological boundary conditions.
\end{definition}

\begin{definition}[Boundary Reflection]
Fix a conditioning boundary $B$ and a parametrization $\lambda$ with boundary label $\lambda_B$.
Define the boundary-reflection map $\mathcal{R}_B:\mathcal{H}\to\mathcal{H}$ to be a reversal
about $B$ in the chosen parametrization, i.e.\ the admissible history obtained by composing
time-reversal $\mathcal{R}$ with the parameter reflection $\lambda\mapsto 2\lambda_B-\lambda$
on the history. In standard physical instantiations, $\mathcal{R}_B$ corresponds to the usual
reversal (e.g.\ momentum inversion / antiunitary conjugation) together with reversal of the
chosen ordering coordinate about the boundary.
\end{definition}


\begin{definition}[Operational Reduction / Projection]
An operational description is obtained via a projection $\Pi:\mathcal{H}\rightarrow\mathcal{D}$,
where $\mathcal{D}$ is a reduced description (effective states, marginals, observables,
or reduced dynamics). The projection $\Pi$ necessarily discards relational information
present in $\mathcal{H}$, is fixed relative to the reduced description, and is blind to
microscopic correlations beyond its resolution. While the representation within
$\mathcal{D}$ (e.g.\ a preferred basis) may depend on dynamical context, the information-discarding content of $\Pi$ does not.
\end{definition}

The claim that the information-discarding content of $\Pi$ is fixed should be
understood in a structural rather than dynamical sense. A projection $\Pi$
defines an equivalence relation over admissible histories, identifying those
differences that are inaccessible to the reduced description. While the
representational form of the reduced variables (e.g., a preferred basis or set
of observables) may depend on dynamical or contextual factors, the equivalence
classes induced by $\Pi$-that is, which distinctions are treated as
operationally irrelevant-remain invariant. Context determines which histories
are realized within these classes, not which distinctions are discarded.
Dynamical laws determine which histories belong to $\mathcal{H}$, whereas the
projection $\Pi$ determines how distinctions among those histories are treated in
the reduced description; these roles are conceptually distinct.

\begin{definition}[Conditioning Boundary]
A conditioning boundary $B$ is the reference relative to which the reduced description
$\Pi(h)$ is defined (preparation/factorization events, coarse-graining anchors, reset or
conditioning protocols). $B$ is not assumed to be a physical beginning or end of time.
This work does not propose a selection principle for $B$; it establishes constraints that
hold given any such choice.
\end{definition}

\begin{definition}[Boundary-Symmetric Projection]\label{def:boundary-symmetric-projection}
The projection $\Pi$ is \emph{boundary-symmetric} (relative to $B$ and $\lambda$) if it does not
distinguish a history from its boundary-reflected counterpart, i.e.
\[
\Pi\circ \mathcal{R}_B = \Pi.
\]
Equivalently, $\Pi$ identifies each history $h$ and its boundary reflection $\mathcal{R}_B(h)$
as operationally indistinguishable in $\mathcal{D}$.
\end{definition}

\noindent
Definition~\ref{def:boundary-symmetric-projection} is not a definition of operational projection in general.
It specifies a verifiable subclass of reductions for which the present constraint applies.
When boundary symmetry fails, arrow-like directedness is classified as asymmetry introduced at the level of the operational projection rather than as a feature of the admissible dynamics.

\begin{remark}[Locus of asymmetry]
Throughout this work, the \emph{locus of asymmetry} refers to the stage at which directional asymmetry enters an account of arrow-like directedness: the admissible dynamics, the operational projection, or the selection of a conditioning boundary.
The theorem constrains how arrows may arise by exhaustively classifying these loci.
\end{remark}

\subsection{Lemmas and Theorem}

By a direction-sensitive diagnostic we mean any function on $\mathcal{D}$ whose values
would distinguish opposite orientations of departure from a conditioning boundary if such
orientation information were available.

\begin{lemma}[Boundary-Reflection Invariance of Reduced Diagnostics]
\label{lem:boundary-invariance}

Let $\mathcal{H}$ be an admissible-history space equipped with a boundary reflection
$\mathcal{R}_B:\mathcal{H}\to\mathcal{H}$ satisfying $\mathcal{R}_B^2=\mathrm{id}$.
Let $\Pi:\mathcal{H}\to\mathcal{D}$ be an operational projection that is
boundary-symmetric, i.e.
\[
\Pi\circ\mathcal{R}_B=\Pi.
\]
Then for any diagnostic quantity $Q:\mathcal{D}\to\mathbb{R}$ defined on the reduced
description, the composed diagnostic $Q\circ\Pi$ is invariant under boundary reflection:
\[
(Q\circ\Pi)(h)=(Q\circ\Pi)(\mathcal{R}_B h)
\qquad\text{for all }h\in\mathcal{H}.
\]
\end{lemma}

\begin{proof}
Since $Q$ is a function on $\mathcal{D}$, it is constant on the equivalence classes
induced by $\Pi$. Boundary symmetry implies $\Pi(h)=\Pi(\mathcal{R}_B h)$ for all
$h\in\mathcal{H}$. Therefore,
\[
(Q\circ\Pi)(h)=Q(\Pi(h))=Q(\Pi(\mathcal{R}_B h))=(Q\circ\Pi)(\mathcal{R}_B h),
\]
establishing boundary-reflection invariance.
\end{proof}

Hereafter, we refer to $\operatorname{sep}(\lambda(h),\lambda_B)$ simply as the
\emph{separation from the conditioning boundary}.

\begin{remark}[Interpretive scope of Lemma~\ref{lem:boundary-invariance}]
Lemma~\ref{lem:boundary-invariance} establishes only invariance under boundary
reflection. It does not, by itself, fix the functional form of reduced diagnostics.
Boundary-radial dependence follows only under the additional empirical assumption
that reduced statistics are operationally indexed by separation from the conditioning
boundary, as made explicit in Corollary~\ref{cor:boundary-radial}.
\end{remark}

\begin{corollary}[Boundary-Radial Representation of Reduced Diagnostics]
\label{cor:boundary-radial}

Assume in addition that the operational protocol indexes reduced diagnostics
by separation from the conditioning boundary, i.e.\ that experimental or
computational statistics are recorded as functions of an unsigned separation
parameter
$\operatorname{sep}(\lambda(h),\lambda_B)\ge 0$ associated with a parametrization
$\lambda$ and boundary value $\lambda_B$.

Under this assumption, any reduced diagnostic admits a boundary-radial representation:
\[
\mathbb{E}[Q\circ\Pi]
=
F\!\left(\operatorname{sep}(\lambda(h),\lambda_B)\right)
\]
for some (not necessarily unique) function $F$ determined by the projection and diagnostic.
\end{corollary}

\begin{remark}
This corollary does not assert that all reduced diagnostics universally take a
boundary-radial form. It states that whenever the reduced description is operationally
indexed by separation from a conditioning boundary-as is standard in thermodynamic,
decoherence, and relaxation contexts-boundary-reflection invariance implies dependence
only on unsigned separation. Diagnostics may still depend on other boundary-invariant
features retained by $\Pi$.
\end{remark}

\begin{remark}[Lemma remark: when boundary-radiality can fail]
Boundary-radial dependence is guaranteed only under boundary-symmetric operational reduction, i.e.\ when $\Pi\circ\mathcal{R}_B=\Pi$.
If a reduced description preserves oriented information, this corresponds to a \emph{verifiable} failure of boundary symmetry:
there exist admissible histories $h$ such that $\Pi(h)\neq \Pi(\mathcal{R}_B(h))$.
Such cases are not dismissed; they are classified as instances where asymmetry is introduced at the level of the operational cut.
The present theorem does not apply to those projections.
\end{remark}

\begin{remark}[Lemma remark: monotonic and oscillatory behavior]
Monotonic behavior (e.g.\ entropy increase) is the special case in which $F$ is monotone.
Oscillatory or cyclic microdynamics may still yield boundary-radial envelopes (e.g.\ decay
of coherence measures) in the reduced description.
\end{remark}

\begin{lemma}[Impossibility of Intrinsic Directionality]
No reduced description $\Pi(\mathcal{H})$ can exhibit a unique intrinsic arrow---direction-dependent
behavior not reducible to boundary separation---unless an additional directional asymmetry
is explicitly introduced (e.g.\ privileging one side of $B$, asymmetric coarse-graining,
restricted parameter domains, or boundary-selection principles).
\end{lemma}

\begin{remark}[Theorem remark: spontaneous symmetry breaking]
If admissible dynamics are symmetric but individual solutions are not, selection of a
particular asymmetric solution (e.g.\ phase selection across quantum or classical phase transitions)
functions as a boundary-selection principle in the present
sense. This does not deny dynamical realization of symmetry breaking; it asserts that
arrow-like directedness attaches to the conditioned description in which a particular branch
or solution is selected, and is therefore boundary-relative rather than intrinsic to
the admissible structure $\mathcal{H}$ itself.
\end{remark}

\begin{theorem}[Inevitability of Projection-Induced Arrows]
\label{thm:projection-induced-arrows}
Given primary admissible dynamics, absence of privileged parametrization, time-reversal
symmetry of the admissible structure, operational reduction via projection, and a
conditioning boundary, any arrow-like behavior in the reduced description is necessarily
a consequence of projection relative to the conditioning boundary and not a property of
the primary dynamics.
\end{theorem}

\begin{remark}[Theorem remark: falsifiability and diagnostic role]
The theorem is not predictive in the sense of yielding novel numerical forecasts,
but it is falsifiable as a structural claim. It would be refuted by the
demonstration of a reduced operational description exhibiting a unique intrinsic
arrow \emph{while} (i) admissible dynamics are symmetric and (ii) operational reduction is
boundary-symmetric in the sense that $\Pi\circ\mathcal{R}_B=\Pi$ holds.
Operationally, boundary symmetry may be assessed by whether boundary-reflected preparations yield indistinguishable reduced statistics under the projection defining $\mathcal{D}$.
Conversely, empirical evidence for
fundamental time-asymmetry would not invalidate the result, but would relocate the
arrow from the projection-induced class to the intrinsic-asymmetry class. In this
way, the framework functions as a diagnostic tool for classifying the source of
observed directedness.
\end{remark}

\begin{remark}[Scope of Explanation]
The theorem constrains the form that arrow explanations must take under the stated premises.
It does not select a unique conditioning boundary, nor explain why a particular boundary is realized.
\end{remark}

\begin{corollary}[Non-Privileging of Time]
No theory satisfying the above assumptions can ontologically privilege time over other
parametrizations. Any such privileging arises from boundary choice, reduction rules,
or interpretive conventions applied after projection.
\end{corollary}

This claim does not deny that particular parametrizations (such as proper time in
general relativity) may be dynamically natural or geometrically distinguished
within specific theories; it denies that such parametrizations alone can ground
intrinsic arrow-like directedness absent additional asymmetric structure. Geometric distinction of timelike directions by itself does not supply an oriented arrow without an accompanying asymmetry or boundary-relative conditioning.