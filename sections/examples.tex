\section{Examples and Applications}
\label{sec:examples}

The following examples are not proposed as new explanations of irreversibility,
but as illustrations of how the constraint theorem manifests across domains in
which the underlying admissible dynamics are widely regarded as symmetric. In
each case, arrow-like behavior arises only after operational reduction anchored
at a conditioning boundary.

The purpose of these examples is not to introduce new mechanisms or results, but
to demonstrate the uniform diagnostic role of conditioning boundaries across
domains already known to admit symmetric admissible dynamics.

The diagnostic structure of the framework may be summarized schematically as follows.

\begin{figure}[t]
\centering
% \resizebox{\columnwidth}{!}{% <--- Uncomment this line to force-fit if still too wide
\begin{tikzpicture}[
  font=\footnotesize,
  % 1. Reduced node distance (vertical and horizontal)
  node distance=1.0cm and 0.4cm, 
  >={Stealth[length=2.0mm]},
  decision/.style={
    draw, diamond, aspect=1.2,
    align=center,
    inner sep=1pt,
    % 2. Reduced width to fixed cm or smaller fraction (0.22 vs 0.30)
    text width=3.5cm 
  },
  outcome/.style={
    draw, rounded corners,
    align=left,
    inner sep=3pt,
    % 2. Reduced width to fixed cm or smaller fraction
    text width=3.5cm 
  },
  edge/.style={->, line width=0.6pt} % Slightly thicker for visibility
]

% Root Node
\node[decision] (dynamics)
{Are the admissible dynamics\\time-reversal symmetric?};

% Level 1 Branches
% Use shifting to manually fine-tune if 'below left' is too aggressive
\node[outcome, below left=of dynamics, xshift=-0.5cm] (intrinsic)
{Arrow may be intrinsic\\(separate dynamical explanation).};

\node[decision, below right=of dynamics, xshift=0.5cm] (projection)
{Is the operational projection\\boundary-symmetric?\\$\Pi \circ \mathcal{R}_B = \Pi$};

% Level 2 Branches (Children of Projection)
\node[outcome, below left=of projection, xshift=0.5cm] (cut)
{Asymmetry at the operational cut\\(arrow is cut-induced).};

\node[outcome, below right=of projection, xshift=-0.5cm] (radial)
{Arrow must be boundary-radial\\(depends only on separation).};

% Edges
\draw[edge] (dynamics) -- node[midway, above left] {No} (intrinsic.north);
\draw[edge] (dynamics) -- node[midway, above right] {Yes} (projection.north);

\draw[edge] (projection) -- node[midway, above left] {No} (cut.north);
\draw[edge] (projection) -- node[midway, above right] {Yes} (radial.north);

\end{tikzpicture}
\caption{Diagnostic decision procedure implied by Theorem~\ref{thm:projection-induced-arrows}.
The framework classifies arrow-like directedness by separating intrinsic dynamical asymmetry from projection-induced asymmetry.
When admissible dynamics are symmetric and the operational reduction is boundary-symmetric, any arrow-like diagnostic in the reduced description must be boundary-radial.
Adapted from Theorem~\ref{thm:projection-induced-arrows} and the accompanying remarks on boundary symmetry.}
\label{fig:decision-tree}
\end{figure}

The following worked examples are included solely to operationalize the diagnostic criterion
\(\Pi\circ \mathcal{R}_B=\Pi\) and to illustrate, in concrete terms, the distinction between
projection-induced boundary-radial arrows and cut-introduced asymmetry.
They are not proposed as new mechanisms for any specific arrow, nor as a boundary-selection principle.

\subsection{Worked diagnostic examples}
\label{subsec:worked-diagnostic-examples}

\subsubsection{Example A (Boundary-symmetric projection): coarse-grained macrostate in reversible Hamiltonian dynamics}
\label{sec:exA-thermo}

\paragraph{Setup.}
Consider a classical Hamiltonian system with microstate \(\Gamma\in\Omega\) (phase space),
Hamiltonian flow \(\Phi_t:\Omega\to\Omega\), and time-reversal involution
\(\mathcal{T}:\Omega\to\Omega\) (e.g.\ momentum inversion) satisfying the standard reversibility relation
\begin{equation}
\mathcal{T}\circ \Phi_t = \Phi_{-t}\circ \mathcal{T}.
\label{eq:ham-reversal}
\end{equation}
Let the admissible-history space \(\mathcal{H}\) be the set of trajectories
\(h:\mathbb{R}\to\Omega\) with \(h(t)=\Phi_t(\Gamma_0)\) for admissible initial data \(\Gamma_0\).
Fix a conditioning boundary \(B\) at parameter value \(\lambda_B=0\), and for this example take
\(\lambda=t\) (the theorem does not require this choice, but it is convenient for exposition).

Define the boundary-reflection map on histories by
\begin{equation}
(\mathcal{R}_B h)(t) \;=\; \mathcal{T}\big(h(-t)\big),
\label{eq:RB-ham}
\end{equation}
i.e.\ reflection about \(t=0\) together with microphysical time reversal.

\paragraph{Operational projection.}
Let \(\Pi:\mathcal{H}\to\mathcal{D}\) be the coarse-graining map that sends a microtrajectory
to a macrotrajectory built from a \emph{time-local} macrostate \(M(t)\) defined by a partition
of phase space into macrocells \(\{C_\alpha\}\). Concretely, define
\begin{equation}
M_h(t) \;:=\; \alpha \quad \text{iff}\quad h(t)\in C_\alpha,
\qquad
\Pi(h) \;:=\; \big(M_h(t)\big)_{t\in\mathbb{R}}.
\label{eq:macro-proj}
\end{equation}
Assume the macrocells are defined by time-reversal even constraints
(e.g.\ densities, energies, coarse position bins), so that each macrocell is invariant under \(\mathcal{T}\):
\begin{equation}
\mathcal{T}(C_\alpha)=C_\alpha\quad \text{for all }\alpha.
\label{eq:cell-invariant}
\end{equation}
This is the standard situation for thermodynamic macrostates: the coarse variables discard
microscopic momentum sign structure, phases, and fine correlations.

\paragraph{Verification of boundary symmetry.}
Using \eqref{eq:RB-ham}--\eqref{eq:cell-invariant}, for any history \(h\) and any \(t\),
\[
(\mathcal{R}_B h)(t)=\mathcal{T}(h(-t))\in C_\alpha
\;\;\Longleftrightarrow\;\;
h(-t)\in \mathcal{T}^{-1}(C_\alpha)=C_\alpha.
\]
Thus \(M_{\mathcal{R}_B h}(t)=M_h(-t)\). If the reduced description \(\mathcal{D}\) is taken to encode
macro-quantities as \emph{functions of unsigned separation from the boundary} (as is typical when conditioning
is specified by a preparation at \(t=0\) without recording an absolute orientation tag), then the induced reduced
macrotrajectory is invariant under boundary reflection. One operationally natural way to state this is:
\begin{equation}
\Pi(h)\equiv \Pi(\mathcal{R}_B h)\quad \text{as reduced descriptions anchored at }B,
\label{eq:PiRB-thermo}
\end{equation}
because the coarse macro-information available at separation \(|t|\) from the preparation boundary is identical
for \(h\) and \(\mathcal{R}_B h\) once oriented departure is not part of the recorded reduced state.\footnote{Equivalently:
the experimental macro-protocol reports reduced statistics conditioned only on the preparation event at \(B\),
and does not include an additional variable that labels ``which side'' of the boundary is being traversed.}

This is precisely the boundary-symmetric case of Definition~\ref{def:boundary-symmetric-projection}:
the operational cut does not retain oriented boundary information.

\paragraph{Boundary-radial diagnostics.}
Let \(Q:\mathcal{D}\to\mathbb{R}\) be a direction-sensitive macro-diagnostic (e.g.\ coarse-grained entropy,
distance-to-equilibrium, or a mixing functional). Under \(\Pi\circ\mathcal{R}_B=\Pi\), Lemma~1 applies, so
\begin{equation}
\mathbb{E}[Q\circ \Pi]\;=\;F\!\left(\operatorname{sep}(t,0)\right)=F(|t|),
\label{eq:Qr-thermo}
\end{equation}
i.e.\ the diagnostic can vary only with separation from the conditioning boundary. In particular, the special
case of monotone \(F\) corresponds to familiar one-sided ``relaxation'' narratives once an orientation convention
is imposed externally, but the theorem itself requires only the boundary-radial dependence.

\paragraph{Operational test.}
This case admits a direct operational check: prepare the same macro-boundary condition at \(B\) and compare
reduced statistics at equal separations on either side of the boundary under a protocol that does not record
an orientation label. Boundary symmetry corresponds to indistinguishability of the reduced statistics for
boundary-reflected preparations at equal \(\operatorname{sep}(t,0)=|t|\). Failure of such indistinguishability constitutes empirical evidence that the projection is boundary-asymmetric in the sense of Definition~\ref{def:boundary-symmetric-projection}.

\subsubsection{Example B (Boundary-asymmetric projection): retaining an oriented boundary label in the reduced description}
\label{sec:exB-asym}

\paragraph{Purpose.}
This example shows how boundary symmetry can fail \emph{even when admissible dynamics remain symmetric},
and how such failure is classified by the framework as asymmetry introduced at the level of the operational cut.
The construction is intentionally simple: the reduced description retains an explicit oriented boundary label.

\paragraph{Setup.}
Let the admissible dynamics and boundary reflection \(\mathcal{R}_B\) be as in Example~A (or any other symmetric
admissible-history space). Fix the same conditioning boundary \(B\) at \(\lambda_B=0\).

\paragraph{Boundary-asymmetric projection.}
Define a reduced description \(\mathcal{D}'\) that contains both the coarse macrostate \(M_h(t)\) and an explicit
\emph{side-of-boundary tag} \(\sigma(t)\in\{+1,-1\}\) indicating the orientation relative to \(B\):
\begin{equation}
\sigma(t):=\operatorname{sgn}(t)
\quad (\sigma(0):=0\ \text{by convention}),\qquad
\Pi'(h)\;:=\;\big(M_h(t),\sigma(t)\big)_{t\in\mathbb{R}}.
\label{eq:Pi-prime}
\end{equation}
Then, for \(t\neq 0\),
\[
\Pi'(h)\neq \Pi'(\mathcal{R}_B h),
\]
because \(\sigma(t)\) flips sign under boundary reflection: \(\sigma(t)\mapsto \sigma(-t)=-\sigma(t)\).
Thus
\begin{equation}
\Pi'\circ \mathcal{R}_B \neq \Pi',
\label{eq:PiRB-fails}
\end{equation}
so boundary symmetry fails \emph{by construction at the level of the reduced description}.

\paragraph{Classification.}
This is not a counterexample to Theorem~\ref{thm:projection-induced-arrows}. It is an instance of the failure mode already described in
Remark~1: the operational cut preserves oriented boundary information, so the boundary-symmetric theorem
does not apply. In such cases, any observed directedness can be attributed (in whole or in part) to the
\emph{asymmetric content of the projection itself} rather than to the admissible dynamics.

\paragraph{Are such projections ``standard''?}
In many standard physical reduced descriptions, orientation relative to a conditioning boundary is \emph{not}
encoded as an additional state variable; the reduced description is specified by preparation at \(B\) together
with statistics as a function of separation from that preparation.
However, orientation-labeled reductions like \eqref{eq:Pi-prime} can occur in practice whenever an operational
protocol explicitly records ``before/after'' or ``pre/post'' labels, or when an external clock variable is carried
into the reduced record as a signed offset from the conditioning event.
The framework therefore treats such cases as neither pathological nor disallowed: they are simply outside the
boundary-symmetric class and are classified accordingly.

\paragraph{Operational diagnostic.}
The same operational criterion applies: if two boundary-reflected preparations yield different reduced statistics
because the reduced record retains an orientation tag, then \(\Pi\circ\mathcal{R}_B=\Pi\) fails and the arrow-like
directedness is cut-introduced.

The following domain-specific illustrations can be read as implicit applications of this diagnostic procedure.

\paragraph{Takeaway.}
Examples~\ref{sec:exA-thermo} and \ref{sec:exB-asym} together make the diagnostic procedure explicit:
under symmetric admissible dynamics, boundary-radial arrows follow when the operational projection is
boundary-symmetric, and need not follow when the projection preserves oriented boundary information.

\subsection{Thermodynamic coarse-graining}

Classical statistical mechanics provides canonical instances in which reversible
microscopic dynamics coexist with macroscopic entropy increase. Hamiltonian
phase-space flow preserves volume and admits time-reversed trajectories, yet
coarse-grained descriptions exhibit relaxation toward equilibrium
\cite{Boltzmann1896,Jaynes1957}.

Within the present framework, the arrow does not attach to the microdynamics but
to the operational projection defining macrostates. Conditioning boundaries are
introduced implicitly through preparation procedures, macrostate definitions,
or coarse-graining anchors. Entropy increase then measures separation from such
boundaries under the reduced description, rather than intrinsic temporal flow of
the admissible dynamics.

\subsection{Quantum measurement and decoherence}

In quantum theory, unitary dynamics preserve reversibility at the level of the
global state, while reduced descriptions obtained by tracing out environmental
degrees of freedom display decoherence, relaxation, and apparent irreversibility.
Preparation, factorization, and measurement events naturally serve as conditioning
boundaries for such reduced descriptions
\cite{Zurek2003,Schlosshauer2007}.

From the present perspective, decoherence-induced arrows are boundary-radial:
coherence measures decay as separation from the conditioning boundary increases.
No intrinsic arrow is attributed to the underlying unitary evolution; directedness
appears only after projection discards system--environment correlations beyond
operational resolution.

\subsection{Computational and information-theoretic arrows}

Reversible computation offers a particularly transparent illustration of the
distinction between admissible dynamics and operational arrows. Classical
reversible logic gates (e.g.\ Toffoli or Fredkin gates) and unitary quantum circuits
implement bijective transformations at the admissible level, admitting
time-reversal symmetry in the relevant sense.

Operational arrows arise only when projections are introduced, most notably
through measurement, reset, or erasure. Logical erasure is a paradigmatic
projection: many distinct logical states are mapped to a single standard state,
discarding logical degrees of freedom. Initialization, reset, or measurement
events thereby function as conditioning boundaries.

In this setting, apparent computational directedness---progress from input to
output, information loss, or entropy production in physical implementations---is
boundary-relative. It attaches to the operational cuts required to obtain usable
computational states, not to the reversible core dynamics themselves. Landauer-type
results connect such projections to thermodynamic costs, but the structural origin
of the arrow lies in the projection and its boundary, not in the admissible logic
or unitary evolution \cite{Landauer1961}.

\subsection{General diagnostic role of boundaries}

Across these examples, the role of the conditioning boundary is uniform. Changing
the boundary---for example, by post-selection, reset, or alternative preparation
protocols---changes the apparent arrow without modifying the underlying admissible
structure. This highlights the diagnostic utility of the framework: whenever
arrow-like behavior is observed in a reduced description governed by symmetric
admissible dynamics, the source of directedness must be traced to the operational
projection and its anchoring boundary.