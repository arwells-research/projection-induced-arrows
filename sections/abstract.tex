We present a minimal constraint theorem that sharply restricts the form arrow-like
directedness can take in physical and informational descriptions. We assume that
(i) admissible histories are ontologically primary and time-reversal symmetric,
(ii) no parametrization is ontologically privileged, and
(iii) operational descriptions require information-losing projection anchored at
a conditioning boundary. Under these conditions, any direction-sensitive
diagnostic defined on the reduced description must depend exclusively on unsigned
separation from that boundary.

Apparent arrows, such as entropy increase, relaxation, decoherence, or
irreversibility, are therefore structurally unavoidable yet non-intrinsic
features of reduced descriptions, rather than properties of the underlying
admissible structure.

The result is formulated as a conditional, falsifiable constraint theorem rather
than as a mechanism for any particular arrow. Projections that preserve oriented
boundary information fall outside the theorem's scope and are classified as
introducing asymmetry at the level of the operational cut. The framework does not
select a unique conditioning boundary or explain why a particular boundary is
realized; instead, it factorizes arrow explanations into a structural component
governing why arrows appear under projection, and an orthogonal boundary-selection
problem.

The theorem applies uniformly across thermodynamic settings, quantum measurement
and decoherence, and information-theoretic analyses. It provides a diagnostic tool
for classifying the origin of observed arrow-like behavior and rules out entire
classes of intrinsic-arrow explanations under symmetric dynamics, without
privileging time or invoking additional metaphysical assumptions.

As immediate corollaries, the theorem classifies all arrow-like diagnostics by the locus of asymmetry, rules out intrinsic arrows under symmetric assumptions, and yields an operational boundary-reflection test.