\section{Introduction}

Arrow-like directedness appears throughout physics and information theory.
Entropy increases, systems relax, coherence decays, measurements appear
irreversible, and computation proceeds from input to output. At the same time,
the fundamental descriptions underlying these phenomena---classical Hamiltonian
mechanics, unitary quantum dynamics, reversible computation---are typically
treated as time-reversal symmetric at the admissible-dynamics level \cite{Boltzmann1896,Jaynes1957}.
 
Discussions of arrow-like directedness in physics typically conflate three distinct questions. The first concerns why direction-sensitive diagnostics appear at all in our descriptions. The second concerns why such diagnostics exhibit a particular orientation (for example, toward what is conventionally called the future). The third concerns why time, rather than some other parametrization, is treated as the relevant dimension along which directedness is assessed.

The present work separates these questions explicitly as follows. Theorem~1 addresses the first: given symmetric admissible dynamics and information-losing operational reduction anchored at a conditioning boundary, arrow-like diagnostics are structurally inevitable and must take a boundary-radial form, depending only on unsigned separation from that boundary. The second question---boundary selection---is treated as orthogonal to the present structural result and concerns why a particular conditioning boundary is physically realized. The third dissolves within the framework: no parametrization is logically privileged, and the apparent centrality of time arises from boundary choice and interpretive convention rather than from the admissible dynamics themselves.

This result is a \emph{structural constraint (no-go) theorem}: it excludes entire classes of arrow explanations under symmetric admissible dynamics, rather than proposing a new mechanism or domain-specific model.

This paper takes a deliberately different approach. Rather than proposing a
mechanism for a particular arrow, we ask a more basic structural question: \emph{given
minimal and widely accepted assumptions about admissible dynamics and
operational description, what form must any arrow-like behavior take?} The
result is not a model of the universe, nor an explanation of why a specific arrow
is observed. It is a constraint on what arrows can be, once certain background
assumptions are accepted \cite{Reichenbach1956,Price1996}.

The key move is to treat admissible histories (trajectories) as ontologically
primary and to refuse any privileged parametrization at that level. Operational
descriptions then arise only through projection: the discarding of correlations
and degrees of freedom required to obtain effective states, marginals, or reduced
dynamics. Such projections are necessarily anchored at conditioning boundaries
relative to which the reduced description is defined \cite{Zurek2003}.

Under these premises, we show that direction-sensitive diagnostics on reduced
descriptions---entropy, decoherence measures, relaxation distances, or
informational loss---are constrained to depend only on separation from the
conditioning boundary. Apparent arrows are therefore \emph{boundary-radial}:
they diagnose distance from an operational cut, not intrinsic temporal flow or
fundamental asymmetry. A unique arrow cannot arise unless additional,
non-minimal asymmetries are explicitly introduced, such as privileging one side
of the boundary, restricting parameter domains, or imposing asymmetric
coarse-graining rules.

It is important to emphasize that this result is not a restatement of the claim
that ``arrows come from coarse-graining.'' No notion of time, entropy,
monotonicity, or thermodynamic behavior is assumed at the outset. The assumptions
invoked are structural rather than dynamical, and the theorem constrains the
\emph{form} that any direction-sensitive diagnostic can take under these minimal
conditions, ruling out entire classes of arrow explanations absent additional
asymmetric structure.

These constraints are made explicit in Corollaries~2--4, which classify arrow-like diagnostics by the locus of asymmetry, establish a no-go result for intrinsic arrows under symmetric assumptions, and yield an operational boundary-reflection test.

The contribution of this work is thus not to explain the arrow of time, but to
clarify why arrow-like behavior is unavoidable in reduced descriptions and why
it cannot, under minimal assumptions, be fundamental. By separating admissible
dynamics from operational projection, the framework makes precise the diagnostic sense in which arrows belong to descriptive practice rather than to the underlying structure itself. This constraint applies uniformly across thermodynamic,
quantum-\allowbreak measurement, and computational settings, and is independent of specific
microphysical details.