\section{Assumptions and Scope}

We use a minimal assumption set designed to avoid importing a preferred temporal
ordering or a specific microphysical theory.

\begin{itemize}[leftmargin=*]
\item \textbf{Primary admissible dynamics:} the fundamental object is a set of admissible
histories (trajectories) defined by fixed constraints.
\item \textbf{No privileged parametrization:} any ordering parameter used to traverse
a history is a descriptive choice, not an ontological primitive.
\item \textbf{Local time-reversal symmetry (conditional):} the admissible structure admits
an involutive reversal map under which admissibility is preserved.
\item \textbf{Operational reduction:} empirical/operational descriptions are obtained by
a projection that discards correlations beyond its resolution.
\item \textbf{Conditioning boundary:} reduced descriptions require anchoring relative to
a conditioning reference (preparation, reset, factorization, etc.).
\end{itemize}

\subsection*{Non-assumptions}
We do \emph{not} assume a Past Hypothesis, a block-universe ontology, any specific
microphysics (classical vs.\ quantum), unitarity at the reduced level, or microscopic
monotonicity. The formal core is a conditional constraint theorem: if the primary
admissible dynamics violate time-reversal symmetry, the theorem does not apply as
stated, but instead provides a diagnostic separation between intrinsic dynamical asymmetry
and asymmetry introduced at the level of operational projection.