\section{Discussion, Limits, and Extensions}

\subsection{Diagnostic Consequences of the Projection Constraint}
\label{subsec:diagnostic-consequences}

Theorem~\ref{thm:projection-induced-arrows} functions not as a generative model but as a
\emph{structural no-go constraint} on admissible explanations of arrow-like directedness.
This subsection makes explicit several immediate diagnostic consequences.
These are not additional assumptions or empirical claims, but logical corollaries
that follow directly from the formal core and classify arrow-like diagnostics by the locus of asymmetry, excluding entire classes of arrow explanations.

\begin{corollary}[Arrow-Origin Classification]
Any arrow-like diagnostic defined on a reduced description must originate from
exactly one of the following sources:
\begin{enumerate}
\item asymmetry in the admissible dynamics themselves,
\item asymmetry in the operational projection,
\item asymmetry introduced by boundary selection.
\end{enumerate}
No arrow-like directedness can arise intrinsically from symmetric admissible dynamics
under boundary-symmetric projection.
\end{corollary}

\begin{remark}
This corollary provides a classification scheme rather than a mechanism.
It partitions arrow explanations by the locus at which asymmetry enters,
independent of the physical domain under consideration.
\end{remark}

\begin{remark}[Forced disclosure]
The structural strictness of the theorem may appear tautological: if direction is excluded
from the projection, it cannot appear in the diagnostic.
This apparent tautology is precisely the diagnostic content of the result.
It compels any account of arrow-like behavior to explicitly identify the specific locus of
asymmetry it relies upon-whether in the admissible dynamics, the operational cut,
or the selection of a conditioning boundary-rather than allowing such asymmetry
to remain implicit or disguised as an emergent feature of symmetric laws.
\end{remark}

\begin{corollary}[Structural Impossibility of Intrinsic Arrows]
In the absence of asymmetry in admissible dynamics, projection, or boundary selection,
direction-sensitive diagnostics on the reduced description are structurally impossible.
\end{corollary}

\begin{remark}
This is a no-go constraint in the standard sense:
a single counterexample---i.e.\ a reduced diagnostic that violates boundary-radial
dependence while satisfying the assumptions of the theorem---would falsify the result.
The corollary does not assert that intrinsic arrows never occur, only that their
existence requires explicit asymmetry.
\end{remark}

\begin{corollary}[Boundary-Reflection Test]
\label{cor:boundary-reflection-test}
If a proposed arrow diagnostic is invariant under boundary reflection at the level of
the reduced description, then it is necessarily projection-induced and boundary-radial.
Conversely, any diagnostic that depends on the sign of separation from the boundary
must violate at least one of the theorem’s assumptions.
\end{corollary}

\begin{remark}
This corollary provides an operational test:
it determines whether an observed arrow reflects intrinsic directional structure
or is an inevitable artifact of projection and conditioning.
The theorem therefore constrains \emph{form}, not magnitude or dynamics.
\end{remark}

\begin{remark}[On Falsifiability]
Like other structural constraints (e.g.\ no-cloning or Bell-type inequalities),
the present theorem is falsifiable without being numerically predictive.
It restricts what forms arrow-like explanations may take, rather than predicting
specific values or time evolutions.
\end{remark}

\noindent
Corollary~\ref{cor:boundary-reflection-test} is operationalized explicitly in
Examples~\ref{sec:exA-thermo} and~\ref{sec:exB-asym}, where boundary-reflected preparations
yield indistinguishable reduced statistics under boundary-symmetric projection (Example~A),
and distinguishable statistics when oriented boundary information is retained in the reduced description (Example~B).
See Sec.~\ref{subsec:worked-diagnostic-examples} for the concrete operational construction.

Concrete operational realizations of this test are summarized in Box 1.

\paragraph{Falsifiability.}
Although the present result is a structural constraint rather than a predictive model, it is falsifiable in the standard sense applicable to no-go theorems. The theorem would be falsified by the exhibition of all three of the following conditions simultaneously:
\begin{enumerate}
\item \emph{Symmetric admissible dynamics:} an admissible-history space $\mathcal{H}$ equipped with a boundary reflection $\mathcal{R}_B$ such that $\mathcal{R}_B(h)\in\mathcal{H}$ for all $h\in\mathcal{H}$ and $\mathcal{R}_B^2=\mathrm{id}$;
\item \emph{Boundary-symmetric operational reduction:} an operational projection $\Pi:\mathcal{H}\to\mathcal{D}$ satisfying $\Pi\circ\mathcal{R}_B=\Pi$;
\item \emph{Oriented reduced diagnostics:} a diagnostic quantity $Q:\mathcal{D}\to\mathbb{R}$ whose expectation value depends on the \emph{sign} of departure from the conditioning boundary, i.e.\ on $\operatorname{sgn}(\lambda(h)-\lambda_B)$,
+where $\operatorname{sgn}$ denotes the sign of the oriented departure,
where $\operatorname{sgn}$ denotes the sign of the oriented departure rather than solely on the unsigned separation $\operatorname{sep}(\lambda(h),\lambda_B)$.
\end{enumerate}
The existence of such a case would demonstrate intrinsically oriented arrow-like behavior emerging from symmetric admissible dynamics under boundary-symmetric operational reduction, thereby directly contradicting Theorem~\ref{thm:projection-induced-arrows}. To our knowledge, no such counterexample has been identified in thermodynamic, quantum, or computational settings.

\paragraph{Interpretive remark.}
This falsifiability criterion clarifies the empirical content of the framework: the theorem does not preclude the existence of arrows, but restricts the circumstances under which they may arise. Any observed direction-sensitive diagnostic must therefore be traceable to a verifiable failure of at least one of the stated assumptions-through intrinsic dynamical asymmetry, boundary-asymmetric projection, or explicit boundary-selection principles.

\subsection{Diagnostic Application: Boltzmann’s H-Theorem}
\label{sec:diagnostic-h-theorem}

Textbook treatments of the Boltzmann H-theorem commonly present entropy increase as a
consequence of purely mechanical dynamics. In a standard formulation, the Boltzmann
equation-together with Liouville’s theorem-is taken to imply monotonic decrease of the
functional $H=\int f\ln f\,d\Gamma$, thereby demonstrating irreversible approach to
equilibrium despite underlying time-reversal symmetric microdynamics
(e.g.\ \citep[Sec.~1.4]{Pathria2011}).

At face value, this presentation suggests that arrow-like directedness emerges from
symmetric admissible dynamics through statistical reasoning alone. The present framework
allows this claim to be examined diagnostically by identifying the locus at which
directional asymmetry enters.

\paragraph{Admissible dynamics and projection.}
At the microscopic level, the admissible dynamics are Hamiltonian and time-reversal
symmetric: for every admissible trajectory in phase space, the momentum-reversed
trajectory is also admissible. The reduced description is provided by the single-particle
distribution function $f(\mathbf{r},\mathbf{p},t)$, obtained by projecting the full
$N$-particle distribution onto one-particle marginals. This projection discards
higher-order correlations and is fixed relative to the reduced description.

\paragraph{Implicit boundary asymmetry.}
A crucial ingredient in the derivation of the Boltzmann equation is the molecular chaos
assumption (Stosszahlansatz), which asserts that particle velocities are uncorrelated
\emph{prior} to collision. This assumption is imposed at a chosen reference time and
propagated forward under the kinetic equation. Importantly, no corresponding assumption is
made about the absence of correlations after collision.

From the standpoint of admissible histories, this asymmetry does not arise from the
Hamiltonian dynamics themselves, nor from the coarse-graining projection as such. Rather,
it reflects a privileged treatment of correlations relative to a conditioning boundary:
correlations are constrained on one side of the boundary (before collision or at an
initial reference time) but not on the other.

\paragraph{Boundary-reflection diagnostic.}
Applying the operational test articulated in Sec.~\ref{subsec:diagnostic-consequences}, one
may ask whether boundary-reflected preparations-obtained by reversing momenta and
correlations at the conditioning boundary-yield indistinguishable reduced statistics under
the same projection. In the presence of the molecular chaos assumption, they do not: the
Boltzmann equation itself fails to be invariant under such reflection because the
assumption singles out a direction of correlation propagation.

Accordingly, the condition $\Pi\circ\mathcal{R}_B=\Pi$ does not hold for the kinetic
description used in the H-theorem derivation. The arrow-like behavior diagnosed by the
monotonic decrease of $H$ therefore does not arise from boundary-symmetric projection acting
on symmetric admissible dynamics.

\paragraph{Classification.}
Within the present taxonomy, the H-theorem exemplifies a case of \emph{boundary-selection
asymmetry}. The entropy increase it describes is boundary-radial relative to a specific
conditioning choice-low correlations at the reference boundary-but this boundary is not
fixed or selected by the mechanical dynamics alone. Claims that the H-theorem derives
irreversibility purely from symmetric mechanics must therefore be understood as incomplete:
a boundary-selection principle is structurally required.

\paragraph{Corrected statement.}
A formulation consistent with the present framework is the following: given symmetric
Hamiltonian dynamics and a boundary condition imposing low correlations at a reference
boundary, entropy increases away from that boundary under the Boltzmann equation. The arrow
thus reflects the imposed boundary asymmetry rather than an intrinsic directionality of the
admissible dynamics.

\begin{center}
\fbox{
\begin{minipage}{0.95\linewidth}
\textbf{Box 1: Operational Tests for Boundary-Symmetric Projection}

\vspace{0.5em}

The boundary-symmetry condition $\Pi \circ \mathcal{R}_B = \Pi$ is not a matter of
representation, but of operational indistinguishability. The following table
summarizes concrete experimental protocols for assessing boundary symmetry across
domains.

\vspace{0.5em}

\newcolumntype{Y}{>{\raggedright\arraybackslash}X}
\newcolumntype{P}[1]{>{\raggedright\arraybackslash}p{#1}}
\begin{tabularx}{\linewidth}{|P{2.9cm}|P{2.9cm}|Y|Y|}
\hline
\textbf{Domain} &
\textbf{Conditioning Boundary $B$} &
\textbf{Forward Protocol} &
\textbf{Boundary-Reflected Protocol} \\
\hline

Classical Statistical Mechanics &
Preparation at $t=0$ &
Prepare macrostate $M$, evolve to $t=+\tau$, measure macro-variables &
Prepare $M$ with momentum-reversed microstates, evolve to $t=-\tau$, measure macro-variables \\
\hline

Quantum Decoherence &
System--environment factorization &
Prepare $\rho_S\otimes\rho_E$, evolve unitarily, trace over environment &
Prepare time-reversed joint state, evolve under reversed Hamiltonian, trace over environment \\
\hline

Reversible Computation &
Memory initialization &
Initialize register, execute circuit, record output &
Reverse circuit without output recording (output readout itself defines a boundary) \\
\hline
\end{tabularx}

\vspace{0.5em}

If the reduced statistics obtained from the forward and boundary-reflected protocols
are indistinguishable, the operational projection is boundary-symmetric.
Failure of indistinguishability constitutes empirical evidence that oriented boundary
information is retained by the projection.
\end{minipage}
}
\end{center}

\subsection{Boundary Selection as a Structural Constraint}

The present framework does not propose a mechanism for the selection of a
conditioning boundary. This is a deliberate restriction rather than a gap, and
reflects the paper’s aim to isolate structural necessity from contingent physical
or cosmological inputs. The contribution of the theorem is to isolate boundary
selection as a distinct explanatory axis: any account of arrow-like directedness
must either (i) posit intrinsic asymmetry in the admissible dynamics, or (ii)
supply a boundary-selection principle explaining why a particular conditioning
boundary is physically realized. Hybrid accounts that appeal to
projection-induced irreversibility while implicitly privileging a boundary are
thereby rendered conceptually explicit.

In this sense, the framework does not relocate the arrow problem arbitrarily, but
forces a sharp separation between the \emph{origin of directedness} (projection
relative to a boundary) and the \emph{origin of boundary selection}. Clarifying
this separation is itself a form of explanatory progress: it prevents asymmetric
assumptions from being silently imported into accounts that otherwise appeal only
to symmetric dynamics and operational reduction, and it prevents conflation of
structural necessity with contingent physical, cosmological, or pragmatic
conditions.

Related discussions of boundary selection and its explanatory role appear, for example,
in analyses of the Past Hypothesis and in broader treatments of temporal arrows
\cite{Wallace2010,Carroll2010}; the present result is orthogonal to such accounts,
constraining admissible forms of arrow explanations without proposing a selection principle.

Boundary-selection principles-cosmological, dynamical, pragmatic, or observer-
relative-may therefore be introduced as orthogonal inputs without altering the
structural result established here. Such principles determine which boundary is
realized; they do not, by themselves, explain why reduced descriptions exhibit
arrow-like diagnostics once conditioning is imposed.

\subsection{Limits of applicability}

The results of this paper rely on two core assumptions: symmetric admissible
dynamics and information-losing operational reduction. If primary admissible
dynamics violate time-reversal symmetry, the theorem does not apply as stated.
In that regime, the framework nevertheless retains diagnostic value by sharply
separating intrinsic asymmetry in the dynamics from projection-induced
asymmetry in reduced descriptions.

Known violations of time-reversal symmetry in the weak interaction (e.g.\ neutral meson systems)
are classified within this framework as \emph{intrinsic dynamical asymmetries}.
Their existence constitutes a failure of the symmetry assumption for that specific sector,
but does not alter the present constraint on arrows arising in effective sectors
(thermodynamic, computational) where admissible dynamics are symmetric.

Similarly, if an operational description preserves directional information by
construction-by encoding asymmetry directly into the operational projection
itself-then boundary-radial behavior need not follow. Such cases fall outside the
scope of the present analysis, which is restricted to projections that treat
admissible histories symmetrically relative to the conditioning boundary.

By contrast, objective collapse models (e.g.\ GRW, Penrose) fall outside the boundary-symmetric
class considered here by positing intrinsically time-asymmetric, non-unitary admissible dynamics.
They are thereby classified as introducing arrows at the dynamical level (Type~1),
distinct from the projection-induced arrows (Type~2) that characterize standard quantum
measurement and decoherence accounts.

\subsection{Extensions and future directions}

Potential extensions of the framework follow naturally from its diagnostic
character: once arrow-like quantities are understood as measures of separation
from a conditioning boundary, information-theoretic and geometric generalizations
become especially natural. Such extensions include explicit
formulations of diagnostic quantities in information-theoretic terms (e.g.\ mutual information relative to a conditioning boundary), as well as applications to 
gravitational or causal horizons treated as operational boundaries. These directions may be pursued without relaxing the minimal assumptions that underwrite the present constraint theorem.

We note that if the global spacetime manifold itself lacks a reflection symmetry
due to topological constraints-e.g.\ differing initial and final singularity structures-
this constitutes an intrinsic asymmetry of the admissible histories, placing it
outside the boundary-symmetric class constrained here.