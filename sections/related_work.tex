\section{Relation to Prior Work}

The emergence of arrow-like directedness from time-reversal symmetric descriptions
has a long history in statistical mechanics, quantum foundations, and
information theory. The present work is aligned with these traditions in spirit,
but differs in emphasis by stating an explicit structural constraint. Under
minimal symmetry and reduction assumptions, direction-sensitive diagnostics in reduced descriptions must be boundary-radial,
i.e.\ functions only of separation from an operational conditioning boundary.

Early treatments by Boltzmann established that thermodynamic irreversibility does
not arise from microscopic laws themselves, but from coarse-grained descriptions
of phase space \cite{Boltzmann1896}. Subsequent developments emphasized the role
of typicality and macrostates, while leaving open the question of why a particular
temporal orientation is selected.

Information-theoretic approaches, particularly those associated with Jaynes,
recast statistical mechanics as inference under constraints, locating entropy
increase in the updating of incomplete descriptions rather than in fundamental
dynamics \cite{Jaynes1957}. The present framework is compatible with this view,
and formalizes the role of conditioning boundaries implicit in such updates.

In quantum theory, environment-induced decoherence and measurement models explain
the appearance of irreversibility through the tracing out of environmental degrees
of freedom \cite{Zurek2003}. Here, preparation, factorization, and measurement
events naturally function as conditioning boundaries, while the underlying unitary
dynamics remain symmetric. Related perspectival and relational approaches emphasize
that physical states and properties are defined relative to systems or observers
\cite{Rovelli1996}.

Recent philosophical analyses have clarified the distinction between fundamental
time symmetry and emergent arrows, often invoking cosmological boundary conditions
such as the Past Hypothesis \cite{Price1996}. The present result does
not compete with such proposals, but instead constrains the form that any arrow
explanation must take once operational reduction is acknowledged, independently
of how a particular boundary is selected.

Finally, Landauer’s principle highlights the connection between logical
irreversibility and physical entropy production, identifying erasure as a
many-to-one mapping at the informational level \cite{Landauer1961}. In the present
framework, logical erasure is a paradigmatic projection, and its associated arrow is boundary-relative to the erasure/reset operation rather than intrinsic to the reversible computational core.

Overall, this work may be viewed as complementary to existing thermodynamic,
quantum, and information-\allowbreak theoretic accounts: it does not propose a new mechanism
for irreversibility, but articulates a general constraint on how arrow-like
directedness can arise in reduced descriptions of symmetric admissible dynamics.